\documentclass{article}
\usepackage{tikz}
\usepackage{amssymb, dsfont}
\usepackage[margin=1in]{geometry}
\usepackage{polski}
\usepackage[utf8]{inputenc}
\newcommand\codex[2][]{
    \tikz[baseline=(s.base)]{
        \node(s)[
            rounded corners,
            fill=blue!5,        % background color
            draw=gray,          % border of box
            text=gray!50!black, % text color
            inner xsep =3pt,    % horizontal space between text and border
            inner ysep =0pt,    % vertical space between text and border
            text height=2ex,    % height of box
            text depth =1ex,    % depth of box
            #1                  % other options
        ]{\texttt{#2}};
    }
}
\usepackage{fancyhdr}
 
\pagestyle{fancy}
\fancyhf{}
\rhead{Raport PRI}
\lhead{Mikołaj Florkiewicz}
\rfoot{\thepage}
 

\title{Projekt PRI}
\author{Mikołaj Florkiewicz}
\date{2015-01-26}
\begin{document}
\maketitle

\section{Interfejs użytkownika}
Program zaimplementowany przy użyciu biblioteki ncurses, dzięki temu 
interfejs jest przejżysty a obsługa go powinna być mozliwa
na wiekszości dostępnych emulatorów terminala.

\section{Implementacja wewnętrzna rekordów}
Rekordy przechowywane są wewnątrz jako płaska struktura - jedyną jej dynamiczną 
częscią jest pole char*, co jednocześnie pozwala przechowywać dowolnie długie 
stringi w dowolnej ilości jak i upraszcza serializację do pliku. 
Poza tym polem w strukturze są pola typu int w których przechowywany jesy ofset 
interesującego nas pola w polu data. 
\section{Serializacja}
Serializacja do pliku jest niezależna od struktury rekordu - może on przechowywać
dowolne dane, jedyne czego potrzebuje to jego wielość, jedynym niedogodnieniem 
jest to, że po przywróceniu sam mechanizm serializacji nie ma możliwości 
sensownego podzielenia danych na pojedyncze rekordy i musi być robione to ręcznie
przez odbiorce danych.
\section{Sortowanie}
Rekordy po załadowaniu są indeksowane, co pozwala oszczedzić każdorazowego sortowania.
Wskażniki na nie przechowywane są w odpowiednim drzewie AVL odpowiadającym 
za sortownie po danym polu. Ze względu na jego właściwosci trawersowanie go, potrzebne
aby zmienić pole po którym sortujemy, jest liniowe od liczby rekordów. Operacje 
dodawnia, usuwania mają złożoność logarytmiczną.



\end{document}
